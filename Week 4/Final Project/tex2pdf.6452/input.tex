\documentclass[]{article}
\usepackage{lmodern}
\usepackage{amssymb,amsmath}
\usepackage{ifxetex,ifluatex}
\usepackage{fixltx2e} % provides \textsubscript
\ifnum 0\ifxetex 1\fi\ifluatex 1\fi=0 % if pdftex
  \usepackage[T1]{fontenc}
  \usepackage[utf8]{inputenc}
\else % if luatex or xelatex
  \ifxetex
    \usepackage{mathspec}
  \else
    \usepackage{fontspec}
  \fi
  \defaultfontfeatures{Ligatures=TeX,Scale=MatchLowercase}
\fi
% use upquote if available, for straight quotes in verbatim environments
\IfFileExists{upquote.sty}{\usepackage{upquote}}{}
% use microtype if available
\IfFileExists{microtype.sty}{%
\usepackage{microtype}
\UseMicrotypeSet[protrusion]{basicmath} % disable protrusion for tt fonts
}{}
\usepackage[margin=1in]{geometry}
\usepackage{hyperref}
\hypersetup{unicode=true,
            pdftitle={Investigate the exponential distribution in R and compare it with the Central Limit Theorem},
            pdfauthor={Pranay Sarkar},
            pdfborder={0 0 0},
            breaklinks=true}
\urlstyle{same}  % don't use monospace font for urls
\usepackage{color}
\usepackage{fancyvrb}
\newcommand{\VerbBar}{|}
\newcommand{\VERB}{\Verb[commandchars=\\\{\}]}
\DefineVerbatimEnvironment{Highlighting}{Verbatim}{commandchars=\\\{\}}
% Add ',fontsize=\small' for more characters per line
\usepackage{framed}
\definecolor{shadecolor}{RGB}{248,248,248}
\newenvironment{Shaded}{\begin{snugshade}}{\end{snugshade}}
\newcommand{\KeywordTok}[1]{\textcolor[rgb]{0.13,0.29,0.53}{\textbf{{#1}}}}
\newcommand{\DataTypeTok}[1]{\textcolor[rgb]{0.13,0.29,0.53}{{#1}}}
\newcommand{\DecValTok}[1]{\textcolor[rgb]{0.00,0.00,0.81}{{#1}}}
\newcommand{\BaseNTok}[1]{\textcolor[rgb]{0.00,0.00,0.81}{{#1}}}
\newcommand{\FloatTok}[1]{\textcolor[rgb]{0.00,0.00,0.81}{{#1}}}
\newcommand{\ConstantTok}[1]{\textcolor[rgb]{0.00,0.00,0.00}{{#1}}}
\newcommand{\CharTok}[1]{\textcolor[rgb]{0.31,0.60,0.02}{{#1}}}
\newcommand{\SpecialCharTok}[1]{\textcolor[rgb]{0.00,0.00,0.00}{{#1}}}
\newcommand{\StringTok}[1]{\textcolor[rgb]{0.31,0.60,0.02}{{#1}}}
\newcommand{\VerbatimStringTok}[1]{\textcolor[rgb]{0.31,0.60,0.02}{{#1}}}
\newcommand{\SpecialStringTok}[1]{\textcolor[rgb]{0.31,0.60,0.02}{{#1}}}
\newcommand{\ImportTok}[1]{{#1}}
\newcommand{\CommentTok}[1]{\textcolor[rgb]{0.56,0.35,0.01}{\textit{{#1}}}}
\newcommand{\DocumentationTok}[1]{\textcolor[rgb]{0.56,0.35,0.01}{\textbf{\textit{{#1}}}}}
\newcommand{\AnnotationTok}[1]{\textcolor[rgb]{0.56,0.35,0.01}{\textbf{\textit{{#1}}}}}
\newcommand{\CommentVarTok}[1]{\textcolor[rgb]{0.56,0.35,0.01}{\textbf{\textit{{#1}}}}}
\newcommand{\OtherTok}[1]{\textcolor[rgb]{0.56,0.35,0.01}{{#1}}}
\newcommand{\FunctionTok}[1]{\textcolor[rgb]{0.00,0.00,0.00}{{#1}}}
\newcommand{\VariableTok}[1]{\textcolor[rgb]{0.00,0.00,0.00}{{#1}}}
\newcommand{\ControlFlowTok}[1]{\textcolor[rgb]{0.13,0.29,0.53}{\textbf{{#1}}}}
\newcommand{\OperatorTok}[1]{\textcolor[rgb]{0.81,0.36,0.00}{\textbf{{#1}}}}
\newcommand{\BuiltInTok}[1]{{#1}}
\newcommand{\ExtensionTok}[1]{{#1}}
\newcommand{\PreprocessorTok}[1]{\textcolor[rgb]{0.56,0.35,0.01}{\textit{{#1}}}}
\newcommand{\AttributeTok}[1]{\textcolor[rgb]{0.77,0.63,0.00}{{#1}}}
\newcommand{\RegionMarkerTok}[1]{{#1}}
\newcommand{\InformationTok}[1]{\textcolor[rgb]{0.56,0.35,0.01}{\textbf{\textit{{#1}}}}}
\newcommand{\WarningTok}[1]{\textcolor[rgb]{0.56,0.35,0.01}{\textbf{\textit{{#1}}}}}
\newcommand{\AlertTok}[1]{\textcolor[rgb]{0.94,0.16,0.16}{{#1}}}
\newcommand{\ErrorTok}[1]{\textcolor[rgb]{0.64,0.00,0.00}{\textbf{{#1}}}}
\newcommand{\NormalTok}[1]{{#1}}
\usepackage{graphicx,grffile}
\makeatletter
\def\maxwidth{\ifdim\Gin@nat@width>\linewidth\linewidth\else\Gin@nat@width\fi}
\def\maxheight{\ifdim\Gin@nat@height>\textheight\textheight\else\Gin@nat@height\fi}
\makeatother
% Scale images if necessary, so that they will not overflow the page
% margins by default, and it is still possible to overwrite the defaults
% using explicit options in \includegraphics[width, height, ...]{}
\setkeys{Gin}{width=\maxwidth,height=\maxheight,keepaspectratio}
\IfFileExists{parskip.sty}{%
\usepackage{parskip}
}{% else
\setlength{\parindent}{0pt}
\setlength{\parskip}{6pt plus 2pt minus 1pt}
}
\setlength{\emergencystretch}{3em}  % prevent overfull lines
\providecommand{\tightlist}{%
  \setlength{\itemsep}{0pt}\setlength{\parskip}{0pt}}
\setcounter{secnumdepth}{0}
% Redefines (sub)paragraphs to behave more like sections
\ifx\paragraph\undefined\else
\let\oldparagraph\paragraph
\renewcommand{\paragraph}[1]{\oldparagraph{#1}\mbox{}}
\fi
\ifx\subparagraph\undefined\else
\let\oldsubparagraph\subparagraph
\renewcommand{\subparagraph}[1]{\oldsubparagraph{#1}\mbox{}}
\fi

%%% Use protect on footnotes to avoid problems with footnotes in titles
\let\rmarkdownfootnote\footnote%
\def\footnote{\protect\rmarkdownfootnote}

%%% Change title format to be more compact
\usepackage{titling}

% Create subtitle command for use in maketitle
\newcommand{\subtitle}[1]{
  \posttitle{
    \begin{center}\large#1\end{center}
    }
}

\setlength{\droptitle}{-2em}
  \title{Investigate the exponential distribution in R and compare it with the
Central Limit Theorem}
  \pretitle{\vspace{\droptitle}\centering\huge}
  \posttitle{\par}
  \author{Pranay Sarkar}
  \preauthor{\centering\large\emph}
  \postauthor{\par}
  \predate{\centering\large\emph}
  \postdate{\par}
  \date{9th August, 2017}


\begin{document}
\maketitle

\subsection{Overview}\label{overview}

The main purpose of the data analysis is to investigate the exponential
distribution and then compare it to the Central Limit Theorem (CLT). For
this analysis, the lambda will be set to 0.2 for all of the simulations.
This investigation will compare the distribution of averages of 40
exponentials over 1000 simulations.

\subsection{Simulations}\label{simulations}

Set the simulation variables -\textgreater{} lambda, exponentials, and
seed.

\begin{Shaded}
\begin{Highlighting}[]
\NormalTok{ECHO=}\OtherTok{TRUE}
\KeywordTok{set.seed}\NormalTok{(}\DecValTok{1337}\NormalTok{)}
\NormalTok{lambda =}\StringTok{ }\FloatTok{0.2}
\NormalTok{exponentials =}\StringTok{ }\DecValTok{40}
\end{Highlighting}
\end{Shaded}

Run Simulations with variables -\textgreater{}

\begin{Shaded}
\begin{Highlighting}[]
\NormalTok{simMeans =}\StringTok{ }\OtherTok{NULL}
\NormalTok{for (i in }\DecValTok{1} \NormalTok{:}\StringTok{ }\DecValTok{1000}\NormalTok{) simMeans =}\StringTok{ }\KeywordTok{c}\NormalTok{(simMeans, }\KeywordTok{mean}\NormalTok{(}\KeywordTok{rexp}\NormalTok{(exponentials, lambda)))}
\end{Highlighting}
\end{Shaded}

\subsection{Sample Mean versus Theoretical
Mean}\label{sample-mean-versus-theoretical-mean}

\paragraph{Sample Mean}\label{sample-mean}

Calculating the mean from the simulations with give the sample mean
-\textgreater{}

\begin{Shaded}
\begin{Highlighting}[]
\KeywordTok{mean}\NormalTok{(simMeans)}
\end{Highlighting}
\end{Shaded}

\begin{verbatim}
## [1] 5.055995
\end{verbatim}

\paragraph{Theoretical Mean}\label{theoretical-mean}

The theoretical mean of an exponential distribution is lambda\^{}-1.

\begin{Shaded}
\begin{Highlighting}[]
\NormalTok{lambda^-}\DecValTok{1}
\end{Highlighting}
\end{Shaded}

\begin{verbatim}
## [1] 5
\end{verbatim}

\paragraph{Comparison}\label{comparison}

There is only a little difference between the simulations sample mean
and the exponential distribution theoretical mean.

\begin{Shaded}
\begin{Highlighting}[]
\KeywordTok{abs}\NormalTok{(}\KeywordTok{mean}\NormalTok{(simMeans)-lambda^-}\DecValTok{1}\NormalTok{)}
\end{Highlighting}
\end{Shaded}

\begin{verbatim}
## [1] 0.05599526
\end{verbatim}

\subsection{Sample Variance versus Theoretical
Variance}\label{sample-variance-versus-theoretical-variance}

\paragraph{Sample Variance}\label{sample-variance}

Calculating the variance from the simulation means with give the sample
variance -\textgreater{}

\begin{Shaded}
\begin{Highlighting}[]
\KeywordTok{var}\NormalTok{(simMeans)}
\end{Highlighting}
\end{Shaded}

\begin{verbatim}
## [1] 0.6543703
\end{verbatim}

\paragraph{Theoretical Variance}\label{theoretical-variance}

The theoretical variance of an exponential distribution is (lambda *
sqrt(n))\^{}-2.

\begin{Shaded}
\begin{Highlighting}[]
\NormalTok{(lambda *}\StringTok{ }\KeywordTok{sqrt}\NormalTok{(exponentials))^-}\DecValTok{2}
\end{Highlighting}
\end{Shaded}

\begin{verbatim}
## [1] 0.625
\end{verbatim}

\paragraph{Comparison}\label{comparison-1}

There is only a slight difference between the simulations sample
variance and the exponential distribution theoretical variance.

\begin{Shaded}
\begin{Highlighting}[]
\KeywordTok{abs}\NormalTok{(}\KeywordTok{var}\NormalTok{(simMeans)-(lambda *}\StringTok{ }\KeywordTok{sqrt}\NormalTok{(exponentials))^-}\DecValTok{2}\NormalTok{)}
\end{Highlighting}
\end{Shaded}

\begin{verbatim}
## [1] 0.0293703
\end{verbatim}

\subsection{Distribution}\label{distribution}

This is a density histogram of the 1000 simulations. There is an overlay
with a normal distribution. Which have a mean of lambda\^{}-1 and
standard deviation of (lambda*sqrt(n))\^{}-1, the theoretical normal
distribution for the simulations.

\begin{Shaded}
\begin{Highlighting}[]
\KeywordTok{library}\NormalTok{(ggplot2)}
\KeywordTok{ggplot}\NormalTok{(}\KeywordTok{data.frame}\NormalTok{(}\DataTypeTok{y=}\NormalTok{simMeans), }\KeywordTok{aes}\NormalTok{(}\DataTypeTok{x=}\NormalTok{y)) +}\StringTok{ }
\StringTok{  }\KeywordTok{geom_histogram}\NormalTok{(}\KeywordTok{aes}\NormalTok{(}\DataTypeTok{y=}\NormalTok{..density..), }\DataTypeTok{binwidth=}\FloatTok{0.2}\NormalTok{, }\DataTypeTok{fill=}\StringTok{"#0072B2"}\NormalTok{, }
                 \DataTypeTok{color=}\StringTok{"black"}\NormalTok{) +}
\StringTok{  }\KeywordTok{stat_function}\NormalTok{(}\DataTypeTok{fun=}\NormalTok{dnorm, }\DataTypeTok{arg=}\KeywordTok{list}\NormalTok{(}\DataTypeTok{mean=}\NormalTok{lambda^-}\DecValTok{1}\NormalTok{, }
                                    \DataTypeTok{sd=}\NormalTok{(lambda*}\KeywordTok{sqrt}\NormalTok{(exponentials))^-}\DecValTok{1}\NormalTok{), }
                \DataTypeTok{size=}\DecValTok{2}\NormalTok{) +}
\StringTok{  }\KeywordTok{labs}\NormalTok{(}\DataTypeTok{title=}\StringTok{"Plot of the Simulations"}\NormalTok{, }\DataTypeTok{x=}\StringTok{"Simulation Mean"}\NormalTok{)}
\end{Highlighting}
\end{Shaded}

\begin{verbatim}
## Warning: Ignoring unknown parameters: arg
\end{verbatim}

\includegraphics{exponential_distribution_vs_CLT_files/figure-latex/unnamed-chunk-9-1.pdf}


\end{document}
